\documentclass[12pt, leqno]{article} %% use to set typesize
\usepackage{fancyhdr}
\usepackage[sort&compress]{natbib}
\usepackage[letterpaper=true,colorlinks=true,linkcolor=black]{hyperref}

\usepackage{amsfonts}
\usepackage{amsmath}
\usepackage{amssymb}
\usepackage{color}
\usepackage{tikz}
\usepackage{pgfplots}
\usepackage{listings}
%\usepackage{courier}
%\usepackage[utf8]{inputenc}
%\usepackage[russian]{babel}

\lstdefinelanguage{Julia}%
  {morekeywords={abstract,break,case,catch,const,continue,do,else,elseif,%
      end,export,false,for,function,immutable,import,importall,if,in,%
      macro,module,otherwise,quote,return,switch,true,try,type,typealias,%
      using,while},%
   sensitive=true,%
   alsoother={$},%
   morecomment=[l]\#,%
   morecomment=[n]{\#=}{=\#},%
   morestring=[s]{"}{"},%
   morestring=[m]{'}{'},%
}[keywords,comments,strings]%

\lstset{
  numbers=left,
  basicstyle=\ttfamily\footnotesize,
  numberstyle=\tiny\color{gray},
  stepnumber=1,
  numbersep=10pt,
}

\newcommand{\iu}{\ensuremath{\mathrm{i}}}
\newcommand{\bbR}{\mathbb{R}}
\newcommand{\bbC}{\mathbb{C}}
\newcommand{\calV}{\mathcal{V}}
\newcommand{\calE}{\mathcal{E}}
\newcommand{\calG}{\mathcal{G}}
\newcommand{\calW}{\mathcal{W}}
\newcommand{\calP}{\mathcal{P}}
\newcommand{\macheps}{\epsilon_{\mathrm{mach}}}
\newcommand{\matlab}{\textsc{Matlab}}

\newcommand{\ddiag}{\operatorname{diag}}
\newcommand{\fl}{\operatorname{fl}}
\newcommand{\nnz}{\operatorname{nnz}}
\newcommand{\tr}{\operatorname{tr}}
\renewcommand{\vec}{\operatorname{vec}}

\newcommand{\vertiii}[1]{{\left\vert\kern-0.25ex\left\vert\kern-0.25ex\left\vert #1
    \right\vert\kern-0.25ex\right\vert\kern-0.25ex\right\vert}}
\newcommand{\ip}[2]{\langle #1, #2 \rangle}
\newcommand{\ipx}[2]{\left\langle #1, #2 \right\rangle}
\newcommand{\order}[1]{O( #1 )}

\newcommand{\kron}{\otimes}


\newcommand{\hdr}[1]{
  \pagestyle{fancy}
  \lhead{Bindel, Spring 2020}
  \rhead{Numerical Analysis}
  \fancyfoot{}
  \begin{center}
    {\large{\bf #1}}
  \end{center}
  \lstset{language=Julia,columns=flexible}  
}


\begin{document}

\hdr{2020-02-07}{2020-02-14}

You may (and should) talk about problems with each other and with me,
providing attribution for any good ideas you might get.  Your final
write-up should be your own.

\paragraph*{1: Growing a system}
Suppose $PA = LU$ is given.  Write an $O(n^2)$ time algorithm to
extend the factorization to an LU factorization of
\[
  M = \begin{bmatrix} A & b \\ c^T & d \end{bmatrix}.
\]
Write your code as a function that takes $P$, $L$, $U$, $b$, $c$, and
$d$ as inputs, and returns extended matrices $\bar{P}$, $\bar{L}$, and
$\bar{U}$.  You should verify that your code is correct for random
choices of $A$, $b$, $c$, and $d$ by checking the backward error,
i.e.~$\|\bar{P} M - \bar{L} \bar{U}\|_F / \|M\|_F$ should be small.

\paragraph*{2: Shrinking back}
Suppose $PA = LU$ is given.  Consider the system $Ax = b + r e_i$
where $x_j = 0$; that is, we allow the $i$th equation not to be
satisfied (by an unknown amount $r$), but enforce that $x_j = 0$.
Express this new problem in terms of a bordered system, and use
block elimination to give an $O(n^2)$ code to compute $x$ and $r$.

\paragraph*{3: Iterative refinement and bordering}
The straightforward algorithm in problem 2 runs into stability
problems when $A$ is nearly rank deficient, even if the modified
system is well-conditioned.  Illustrate the problem with the
matrix
\[
  A = \begin{bmatrix} 1 & 1 \\ 1 & 1+\delta \end{bmatrix}
\]
for $\delta = 10^{-12}$.  Show experimentally that a step of iterative
refinement fixes the issue.

\end{document}
