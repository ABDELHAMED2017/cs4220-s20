\documentclass[12pt, leqno]{article} %% use to set typesize
\usepackage{fancyhdr}
\usepackage[sort&compress]{natbib}
\usepackage[letterpaper=true,colorlinks=true,linkcolor=black]{hyperref}

\usepackage{amsfonts}
\usepackage{amsmath}
\usepackage{amssymb}
\usepackage{color}
\usepackage{tikz}
\usepackage{pgfplots}
\usepackage{listings}
%\usepackage{courier}
%\usepackage[utf8]{inputenc}
%\usepackage[russian]{babel}

\lstdefinelanguage{Julia}%
  {morekeywords={abstract,break,case,catch,const,continue,do,else,elseif,%
      end,export,false,for,function,immutable,import,importall,if,in,%
      macro,module,otherwise,quote,return,switch,true,try,type,typealias,%
      using,while},%
   sensitive=true,%
   alsoother={$},%
   morecomment=[l]\#,%
   morecomment=[n]{\#=}{=\#},%
   morestring=[s]{"}{"},%
   morestring=[m]{'}{'},%
}[keywords,comments,strings]%

\lstset{
  numbers=left,
  basicstyle=\ttfamily\footnotesize,
  numberstyle=\tiny\color{gray},
  stepnumber=1,
  numbersep=10pt,
}

\newcommand{\iu}{\ensuremath{\mathrm{i}}}
\newcommand{\bbR}{\mathbb{R}}
\newcommand{\bbC}{\mathbb{C}}
\newcommand{\calV}{\mathcal{V}}
\newcommand{\calE}{\mathcal{E}}
\newcommand{\calG}{\mathcal{G}}
\newcommand{\calW}{\mathcal{W}}
\newcommand{\calP}{\mathcal{P}}
\newcommand{\macheps}{\epsilon_{\mathrm{mach}}}
\newcommand{\matlab}{\textsc{Matlab}}

\newcommand{\ddiag}{\operatorname{diag}}
\newcommand{\fl}{\operatorname{fl}}
\newcommand{\nnz}{\operatorname{nnz}}
\newcommand{\tr}{\operatorname{tr}}
\renewcommand{\vec}{\operatorname{vec}}

\newcommand{\vertiii}[1]{{\left\vert\kern-0.25ex\left\vert\kern-0.25ex\left\vert #1
    \right\vert\kern-0.25ex\right\vert\kern-0.25ex\right\vert}}
\newcommand{\ip}[2]{\langle #1, #2 \rangle}
\newcommand{\ipx}[2]{\left\langle #1, #2 \right\rangle}
\newcommand{\order}[1]{O( #1 )}

\newcommand{\kron}{\otimes}


\newcommand{\hdr}[1]{
  \pagestyle{fancy}
  \lhead{Bindel, Spring 2020}
  \rhead{Numerical Analysis}
  \fancyfoot{}
  \begin{center}
    {\large{\bf #1}}
  \end{center}
  \lstset{language=Julia,columns=flexible}  
}


\begin{document}

\pagestyle{fancy}
\lhead{Bindel, Fall 2020}
\rhead{Numerical Analysis}
\fancyfoot{}
\begin{center}
  {\large{\bf Midterm}} \\ (due: 2020-03-13)
\end{center}
\lstset{language=Julia,columns=flexible}

You may (and should) consult any resources you wish {\em except} for
people (outside the course staff).  Provide attribution for any good
ideas you might get.  Your final write-up should be your own.


\paragraph*{True/false}
For each of the following statements, either give a brief argument
that it is true or provide an example to show it is false.
\begin{itemize}
\item[1 pt]
  The eigenvalues of a real symmetric positive definite matrix are
  all positive.
\item[1 pt]
  Every square nonsingular matrix $A$ can be decomposed as $A =
  LU$ where $L$ is unit lower triangular and $U$ is upper triangular.
\item[1 pt]
  The eigenvalues of a 2-by-2 matrix are differentiable functions
  of the matrix entries.
\item[1 pt]
  For every square matrix, power iteration eventually converges to
  a dominant eigenvector.
\item[1 pt]
  If $A \in \bbR^{n \times n}$ has condition number $\kappa_2(A) = 1$,
  then $A = \alpha Q$ for some real $\alpha \neq 0$
  and orthogonal matrix $Q \in \bbR^{n \times n}$.
\item[1 pt]
  If $A \in \bbR^{m \times n}$ has full column rank, then $A^\dagger A = I$.
\end{itemize}
  
\paragraph*{Speedy and stable}
Consider each of the following:
\begin{itemize}
\item[2 pts]
  Rewrite the following code for better speed and numerical stability
  assuming $PA = LU$ is already computed.
  \begin{lstlisting}
    dx = -inv(A)*dA*inv(A)*b
  \end{lstlisting}
\item[2 pts]
  For $x > 1$, the equation $x = \cosh(y)$ can be solved as
  \[
    y = -\ln\left(x - \sqrt{x^2-1}\right).
  \]
  What happens when $x = 10^8$?  How can we fix it?
\item[2 pts]
  Explain the output of the following code fragment
  \begin{lstlisting}
    x = 2;
    for k = 1:100, x = sqrt(x); end
    for k = 1:100, x = x^2;     end
    disp(x);
  \end{lstlisting}
\end{itemize}


\paragraph*{Polynomial fitting}
Consider fitting a polynomial of degree at most $d$
to $n$ data points, i.e.
\[
  \mbox{minimize } \sum_{j=1}^n (p(x_i)-y_i)^2
\]
\begin{itemize}
\item[2 pts]
  Complete the following code to find $p(x) = \sum_{j=0}^d c_j x^j$
  that solves the minimization problem by a standard
  linear least squares formulation:
  \begin{lstlisting}
    function [c] = poly_least_squares(x, y)
  \end{lstlisting}
  You may write equivalent code in Julia or Python.
\item[2 pts]
  Modify the code to solve the problem
  \[
    \mbox{minimize } \sum_{j=1}^n (p(x_i)-y_i)^2 +
    \mu \|c\|^2
  \]
  In MATLAB, your routine should have the interface
  \begin{lstlisting}
    function [c] = poly_least_squares_tik(x, y, mu)
  \end{lstlisting}
  You may write equivalent code in Julia or Python.
\item[2 pts]
  Modify the code to solve the problem
  \[
    \mbox{minimize } \sum_{j=1}^n (p(x_i)-y_i)^2 +
    \mu \int_{-1}^1 p'(x)^2 \, dx
    \]
  In MATLAB, your routine should have the interface
  \begin{lstlisting}
    function [c] = poly_least_squares_smooth(x, y, mu)
  \end{lstlisting}
  You may write equivalent code in Julia or Python.
\end{itemize}


\paragraph*{Sensitive pseudoinverse}
Consider the problem
\[
  x(s) = (A+sE)^\dagger b
\]
where $A \in \bbR^{m \times n}$ has full column rank.  Suppose
the economy QR factorization $A = QR$ is given.
\begin{itemize}
\item[2 pts]
  Compute $x(0)$.  Your code should take at most $O(mn)$ time.  In MATLAB,
  this means filling in the second line in this fragment:
  \begin{lstlisting}
    [Q,R] = qr(A,0); % This is O(mn^2)
    x0 = ???;        % This should be O(mn)
  \end{lstlisting}
\item[4 pts]
  Compute $x'(0)$ (worth two points).
  Your code should take at most $O(mn)$ time (worth two points).
  You can sanity check your code by a finite difference test:
  \begin{lstlisting}
    h = 1e-6;
    xp = (A+h*E)\b;
    xm = (A-h*E)\b;
    dx_fd = (xp-xm)/2/h;
    dx = % your computation
    relerr = norm(dx_fd-dx)/norm(dx)
  \end{lstlisting}
\end{itemize}


\paragraph*{Conditioning}
\begin{itemize}
\item[2 pts]
  Argue that if $A \in \bbR^{n \times n}$ is symmetric and positive definite and
  $A_{11} \in \bbR^{k \times k}$ is a leading principal submatrix,
  then $\kappa_2(A_{11}) \leq \kappa(A)$, where $\kappa$ denotes
  the usual 2-norm condition number for solving linear systems. \\[2mm]
  {\em Hint}: You may use the {\em interlace theorem}, which tells
  us that if $\mu_1 \leq \mu_2 \leq \ldots \leq \mu_k$ are the
  eigenvalues of $A_{11}$ and 
  $\lambda_1 \leq \lambda_2 \leq \ldots \leq \lambda_n$ are the
  eigenvalues of $A$, then
  \[
    \lambda_j \leq \mu_j \leq \lambda_{n-k+j}.
  \]
\item[2 pts]
  Show by example that the hypothesis that $A$ is positive definite
  is necessary in the above statement.
\item[2 pts]
  Show that if
  $A \in \bbR^{m \times n} = \begin{bmatrix} A_1 & A_2 \end{bmatrix}$,
  then $\kappa(A_1) \leq \kappa(A)$, where $\kappa$ here denotes the
  usual condition number for least squares.
\end{itemize}

\end{document}
