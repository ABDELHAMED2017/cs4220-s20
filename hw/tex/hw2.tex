\documentclass[12pt, leqno]{article} %% use to set typesize
\usepackage{fancyhdr}
\usepackage[sort&compress]{natbib}
\usepackage[letterpaper=true,colorlinks=true,linkcolor=black]{hyperref}

\usepackage{amsfonts}
\usepackage{amsmath}
\usepackage{amssymb}
\usepackage{color}
\usepackage{tikz}
\usepackage{pgfplots}
\usepackage{listings}
%\usepackage{courier}
%\usepackage[utf8]{inputenc}
%\usepackage[russian]{babel}

\lstdefinelanguage{Julia}%
  {morekeywords={abstract,break,case,catch,const,continue,do,else,elseif,%
      end,export,false,for,function,immutable,import,importall,if,in,%
      macro,module,otherwise,quote,return,switch,true,try,type,typealias,%
      using,while},%
   sensitive=true,%
   alsoother={$},%
   morecomment=[l]\#,%
   morecomment=[n]{\#=}{=\#},%
   morestring=[s]{"}{"},%
   morestring=[m]{'}{'},%
}[keywords,comments,strings]%

\lstset{
  numbers=left,
  basicstyle=\ttfamily\footnotesize,
  numberstyle=\tiny\color{gray},
  stepnumber=1,
  numbersep=10pt,
}

\newcommand{\iu}{\ensuremath{\mathrm{i}}}
\newcommand{\bbR}{\mathbb{R}}
\newcommand{\bbC}{\mathbb{C}}
\newcommand{\calV}{\mathcal{V}}
\newcommand{\calE}{\mathcal{E}}
\newcommand{\calG}{\mathcal{G}}
\newcommand{\calW}{\mathcal{W}}
\newcommand{\calP}{\mathcal{P}}
\newcommand{\macheps}{\epsilon_{\mathrm{mach}}}
\newcommand{\matlab}{\textsc{Matlab}}

\newcommand{\ddiag}{\operatorname{diag}}
\newcommand{\fl}{\operatorname{fl}}
\newcommand{\nnz}{\operatorname{nnz}}
\newcommand{\tr}{\operatorname{tr}}
\renewcommand{\vec}{\operatorname{vec}}

\newcommand{\vertiii}[1]{{\left\vert\kern-0.25ex\left\vert\kern-0.25ex\left\vert #1
    \right\vert\kern-0.25ex\right\vert\kern-0.25ex\right\vert}}
\newcommand{\ip}[2]{\langle #1, #2 \rangle}
\newcommand{\ipx}[2]{\left\langle #1, #2 \right\rangle}
\newcommand{\order}[1]{O( #1 )}

\newcommand{\kron}{\otimes}


\newcommand{\hdr}[1]{
  \pagestyle{fancy}
  \lhead{Bindel, Spring 2020}
  \rhead{Numerical Analysis}
  \fancyfoot{}
  \begin{center}
    {\large{\bf #1}}
  \end{center}
  \lstset{language=Julia,columns=flexible}  
}


\begin{document}

\hdr{2020-01-31}{2020-02-07}

You may (and should) talk about problems with each other and with me,
providing attribution for any good ideas you might get.  Your final
write-up should be your own.

% Sensitivity, error analysis, norm manipulations
% Floating point
% Gaussian elimination?

\paragraph*{1: Conditioning}
For a differentiable function $f : \bbR^n \rightarrow \bbR^m$, we
define the (2-norm) condition number at $x$ to be
\[
  \kappa_2(x) = \frac{\|J(x)\|_2 \|x\|_2}{\|f(x)\|_2}
\]
where $J(x)$ is the {\em Jacobian} matrix (the matrix of partial
derivatives).  This generalizes the scalar condition number defined
in the notes.  Given this definition, what is the condition number of
$f : \bbR^2 \rightarrow \bbR$ given by $f(x) = x_1 x_2$?
Under what circumstances is this condition number large?

\paragraph*{2: Box-Cox}
The Box-Cox family of transformations is sometimes used in statistical
applications to normalize non-negative data.  The transformation has
the form
\[
  g_\lambda(x) =
  \begin{cases}
    \frac{x^\lambda-1}{\lambda}, & \lambda \neq 0 \\
    \log x, & \lambda = 0
  \end{cases}
\]
The most obvious implementation (in MATLAB) is
\begin{lstlisting}
  if lambda == 0
    g = log(x);
  else
    g = (x^lambda-1)/lambda;
  end
\end{lstlisting}
Unfortunately, this is prone to large relative errors when
$|\lambda \log x| \ll 1$.  Explain why, and suggest an alternative
with better error in this case.  You may wish to use the function
{\tt expm1} for your solution (though there are other approaches as well).

\paragraph*{3: Pi, see!}
The following routine estimates $\pi$ by recursively computing the
semiperimeter of a sequence of $2^{k+1}$-gons embedded in the unit circle:
\lstset{language=matlab,frame=lines,columns=flexible}
\lstinputlisting{code/pibad.m}
Plot the absolute error $|s_k-\pi|$ against $k$ on a semilog plot.
Explain why the algorithm behaves as it does, and describe a
reformulation of the algorithm that does not suffer from this problem.


\end{document}
