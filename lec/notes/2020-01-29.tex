\documentclass[12pt, leqno]{article}
\usepackage{fancyhdr}
\usepackage[sort&compress]{natbib}
\usepackage[letterpaper=true,colorlinks=true,linkcolor=black]{hyperref}

\usepackage{amsfonts}
\usepackage{amsmath}
\usepackage{amssymb}
\usepackage{color}
\usepackage{tikz}
\usepackage{pgfplots}
\usepackage{listings}
%\usepackage{courier}
%\usepackage[utf8]{inputenc}
%\usepackage[russian]{babel}

\lstdefinelanguage{Julia}%
  {morekeywords={abstract,break,case,catch,const,continue,do,else,elseif,%
      end,export,false,for,function,immutable,import,importall,if,in,%
      macro,module,otherwise,quote,return,switch,true,try,type,typealias,%
      using,while},%
   sensitive=true,%
   alsoother={$},%
   morecomment=[l]\#,%
   morecomment=[n]{\#=}{=\#},%
   morestring=[s]{"}{"},%
   morestring=[m]{'}{'},%
}[keywords,comments,strings]%

\lstset{
  numbers=left,
  basicstyle=\ttfamily\footnotesize,
  numberstyle=\tiny\color{gray},
  stepnumber=1,
  numbersep=10pt,
}

\newcommand{\iu}{\ensuremath{\mathrm{i}}}
\newcommand{\bbR}{\mathbb{R}}
\newcommand{\bbC}{\mathbb{C}}
\newcommand{\calV}{\mathcal{V}}
\newcommand{\calE}{\mathcal{E}}
\newcommand{\calG}{\mathcal{G}}
\newcommand{\calW}{\mathcal{W}}
\newcommand{\calP}{\mathcal{P}}
\newcommand{\macheps}{\epsilon_{\mathrm{mach}}}
\newcommand{\matlab}{\textsc{Matlab}}

\newcommand{\ddiag}{\operatorname{diag}}
\newcommand{\fl}{\operatorname{fl}}
\newcommand{\nnz}{\operatorname{nnz}}
\newcommand{\tr}{\operatorname{tr}}
\renewcommand{\vec}{\operatorname{vec}}

\newcommand{\vertiii}[1]{{\left\vert\kern-0.25ex\left\vert\kern-0.25ex\left\vert #1
    \right\vert\kern-0.25ex\right\vert\kern-0.25ex\right\vert}}
\newcommand{\ip}[2]{\langle #1, #2 \rangle}
\newcommand{\ipx}[2]{\left\langle #1, #2 \right\rangle}
\newcommand{\order}[1]{O( #1 )}

\newcommand{\kron}{\otimes}


\newcommand{\hdr}[1]{
  \pagestyle{fancy}
  \lhead{Bindel, Spring 2020}
  \rhead{Numerical Analysis}
  \fancyfoot{}
  \begin{center}
    {\large{\bf #1}}
  \end{center}
  \lstset{language=Julia,columns=flexible}  
}


\begin{document}
\hdr{2020-01-29}

\section{Norms revisited}

In the last lecture, we discussed norms, including induced norms:
if $A$ maps between two normed vector spaces $\calV$ and $\calW$,
the {\em induced norm} on $A$ is
\[
  \|A\|_{\calV,\calW}
  = \sup_{v \neq 0} \frac{ \|Av\|_{\calW} }{ \|v\|_{\calV} }
  = \sup_{\|v\|_{\calV} = 1} \|Av\|_{\calW}.
\]
When $\calV$ is finite-dimensional (as it always is in this class),
the unit ball $\{v \in \calV : \|v\| = 1\}$ is compact, and $\|Av\|$
is a continuous function of $v$, so the supremum is actually attained.
Induced norms have a number of nice properties, not the least of
which are the submultiplicative properties
\begin{align*}
  \|Av\| & \leq \|A\| \|v\| \\
  \|AB\| & \leq \|A\| \|B\|.
\end{align*}
The first property ($\|Av\| \leq \|A\| \|v\|$) is clear from the
definition of the vector norm.  The second property is almost as easy
to prove:
\begin{align*}
  \|AB\| =    \max_{\|v\| = 1} \|ABv\|
         \leq \max_{\|v\| = 1} \|A\| \|Bv\|
         = \|A\| \|B\|.
\end{align*}

The matrix norms induced when $\calV$ and $\calW$ are supplied with a 1-norm,
2-norm, or $\infty$-norm are simply called the matrix 1-norm, 2-norm,
and $\infty$-norm.  The matrix 1-norm and $\infty$-norm are given by
\begin{align*}
  \|A\|_1       &= \max_{j} \sum_{i} |a_{ij}| \\
  \|A\|_{\infty} &= \max_{i} \sum_{j} |a_{ij}|.
\end{align*}
These norms are nice because they are easy to compute; the two norm
is nice for other reasons, but is not easy to compute.

\subsection{Norms and Neumann series}

We will do a great deal of operator norm manipulation this semester,
almost all of which boils down to repeated use of the triangle
inequality and the submultiplicative property.  For now, we illustrate
the point by a simple, useful example: the matrix version of
the geometric series.

Suppose $F$ is a square matrix such that $\|F\| < 1$ in
some operator norm, and consider the power series
\[
  \sum_{j=0}^n F^j.
\]
Note that $\|F^j\| \leq \|F\|^j$ via the submultiplicative property
of induced operator norms.
By the triangle inequality, the partial sums satisfy
\[
  (I-F) \sum_{j=0}^n F^j = I - F^{n+1}.
\]
Hence, we have that
\[
  \|(I-F) \sum_{j=0}^n F^j - I\| \leq \|F\|^{n+1} \rightarrow 0
  \mbox{ as } n \rightarrow \infty,
\]
i.e. $I-F$ is invertible and the inverse is given by the convergent
power series (the geometric series or {\em Neumann series})
\[
  (I-F)^{-1} = \sum_{j=0}^\infty F^j.
\]
By applying submultiplicativity and triangle inequality to the partial
sums, we also find that
\[
  \|(I-F)^{-1}\| \leq \sum_{j=0}^\infty \|F\|^j = \frac{1}{1-\|F\|}.
\]

Note as a consequence of the above that if $\|A^{-1} E\| < 1$ then
\[
  \|(A+E)^{-1}\|
  = \|(I+A^{-1} E)^{-1} A^{-1}\|
  \leq \frac{\|A^{-1}\|}{1-\|A^{-1} E\|}.
\]
That is, the Neumann series gives us a sense of how a small
perturbation to $A$ can change the norm of $A^{-1}$.

%% \subsection{ The 2-norm}

%% The matrix 2-norm is very useful, but it is also not so straightforward
%% to compute.  Last time, we showed how to think about computing
%% $\|A\|_2$ via the SVD.  We now take a different tack, foreshadowing
%% topics to come in the class.  Depending on timing, I may not talk
%% about this in lecture, but I think it is worth mentioning in the notes.

%% If $A$ is a real matrix, then we have
%% \begin{align*}
%%   \|A\|_2^2
%%     &= \left( \max_{\|v\|_2 = 1} \|Av\| \right)^2 \\
%%     &= \max_{\|v\|_2^2 = 1} \|Av\|^2 \\
%%     &= \max_{v^T v = 1} v^T A^T A v.
%% \end{align*}
%% This is a constrained optimization problem, to which we will apply the
%% method of Lagrange multipliers: that is, we seek critical points for
%% the functional
%% \[
%%   L(v, \mu) = v^T A^T A v - \mu (v^T v-1).
%% \]
%% Differentiate in an arbitrary direction $(\delta v, \delta \mu)$ to find
%% \begin{align*}
%%   2 \delta v^T (A^T A v - \mu v) & = 0, \\
%%   \delta \mu (v^T v-1) & = 0.
%% \end{align*}
%% Therefore, the stationary points satisfy the eigenvalue problem
%% \[
%%   A^T A v = \mu v.
%% \]
%% The eigenvalues of $A^T A$ are non-negative (why?), so we will call
%% them $\sigma_i^2$.  The positive values $\sigma_i$ are exactly the {\em
%%   singular values} of $A$ --- the diagonal elements of the matrix
%% $\Sigma$ in the singular value decomposition from last lecture ---
%% and the eigenvectors of $A^T A$ are the right singular vectors ($V$).

\section{Notions of error}

The art of numerics is finding an approximation with a fast algorithm,
a form that is easy to analyze, and an error bound.  Given a task, we
want to engineer an approximation that is good enough, and that
composes well with other approximations.  To make these goals precise,
we need to define types of errors and error propagation, and some
associated notation -- which is the point of this lecture.

\subsection{Absolute and relative error}

Suppose $\hat{x}$ is an approximation to $x$.
The {\em absolute error} is
\[
  e_{\mathrm{abs}} = |\hat{x}-x|.
\]
Absolute error has the same dimensions as $x$,
and can be misleading without some context.  An error
of one meter per second is dramatic if $x$ is my walking
pace; if $x$ is the speed of light, it is a very small error.

The {\em relative error} is a measure with a more natural
sense of scale:
\[
  e_{\mathrm{rel}} = \frac{|\hat{x}-x|}{|x|}.
\]
Relative error is familiar in everyday life: when someone
talks about an error of a few percent, or says that a given
measurement is good to three significant figures, she is describing
a relative error.

We sometimes estimate the relative error in approximating
$x$ by $\hat{x}$ using the relative error in approximating
$\hat{x}$ by $x$:
\[
  \hat{e}_{\mathrm{rel}} = \frac{|\hat{x}-x|}{|\hat{x}|}.
\]
As long as $\hat{e}_{\mathrm{rel}} < 1$, a little algebra gives that
\[
  \frac{\hat{e}_{\mathrm{rel}}}{1+\hat{e}_{\mathrm{rel}}} \leq
  e_{\mathrm{rel}} \leq
  \frac{\hat{e}_{\mathrm{rel}}}{1-\hat{e}_{\mathrm{rel}}}.
\]
If we know $\hat{e}_{\mathrm{rel}}$ is much less than one, then it
is a good estimate for $e_{\mathrm{rel}}$.  If
$\hat{e}_{\mathrm{rel}}$ is not much less than one,
we know that $\hat{x}$ is a poor approximation to $x$.
Either way, $\hat{e}_{\mathrm{rel}}$ is often just as useful
as $e_{\mathrm{rel}}$, and may be easier to estimate.

Relative error makes no sense for $x = 0$, and may be too pessimistic
when the property of $x$ we care about is ``small enough.''  A natural
intermediate between absolute and relative errors is the mixed error
\[
  e_{\mathrm{mixed}} = \frac{|\hat{x}-x|}{|x| + \tau}
\]
where $\tau$ is some natural scale factor associated with $x$.

\subsection{Errors beyond scalars}

Absolute and relative error make sense for vectors as well as scalars.
If $\| \cdot \|$ is a vector
norm and $\hat{x}$ and $x$ are vectors, then the (normwise) absolute
and relative errors are
\begin{align*}
  e_{\mathrm{abs}} &= \|\hat{x}-x\|, &
  e_{\mathrm{rel}} &= \frac{\|\hat{x}-x\|}{\|x\|}.
\end{align*}
We might also consider the componentwise absolute or relative errors
\begin{align*}
  e_{\mathrm{abs},i} &= |\hat{x}_i-x_i| &
  e_{\mathrm{rel},i} &= \frac{|\hat{x}_i-x_i|}{|x_i|}.
\end{align*}
The two concepts are related: the maximum componentwise relative error
can be computed as a normwise error in a norm defined in terms of the
solution vector:
\begin{align*}
  \max_i e_{\mathrm{rel},i} &= \vertiii{\hat{x}-x}
\end{align*}
where $\vertiii{z} = \|\ddiag(x)^{-1} z\|$.
More generally, absolute error makes sense whenever we can measure
distances between the truth and the approximation; and relative error
makes sense whenever we can additionally measure the size of the
truth.  However, there are often many possible notions of distance
and size; and different ways to measure give different notions of
absolute and relative error.  In practice, this deserves some care.

\subsection{Forward and backward error and conditioning}

We often approximate a function $f$ by another function $\hat{f}$.
For a particular $x$, the {\em forward} (absolute) error is
\[
  |\hat{f}(x)-f(x)|.
\]
In words, forward error is the function {\em output}.  Sometimes,
though, we can think of a slightly wrong {\em input}:
\[
  \hat{f}(x) = f(\hat{x}).
\]
In this case, $|x-\hat{x}|$ is called the {\em backward} error.
An algorithm that always has small backward error is {\em backward stable}.

A {\em condition number} a tight constant relating relative output
error to relative input error.  For example, for the problem of
evaluating a sufficiently nice function $f(x)$ where $x$ is the input
and $\hat{x} = x+h$ is a perturbed input (relative error $|h|/|x|$),
the condition number $\kappa[f(x)]$ is the smallest constant such that
\[
  \frac{|f(x+h)-f(x)|}{|f(x)|} \leq \kappa[f(x)] \frac{|h|}{|x|} + o(|h|)
\]
If $f$ is differentiable, the condition number is
\[
\kappa[f(x)] =
  \lim_{h \neq 0} \frac{|f(x+h)-f(x)|/|f(x)|}{|(x+h)-x|/|x|} =
  \frac{|f'(x)||x|}{|f(x)|}.
\]
If $f$ is Lipschitz in a neighborhood of $x$ (locally Lipschitz), then
\[
\kappa[f(x)] =
  \frac{M_{f(x)}|x|}{|f(x)|}.
\]
where $M_f$ is the smallest constant such that
$|f(x+h)-f(x)| \leq M_f |h| + o(|h|)$.  When the problem has no linear
bound on the output error relative to the input error, we sat the
problem has an {\em infinite} condition number.  An example is
$x^{1/3}$ at $x = 0$.

A problem with a small condition number is called {\em well-conditioned};
a problem with a large condition number is {\em ill-conditioned}.
A backward stable algorithm applied to a well-conditioned problem has
a small forward error.

\section{Perturbing matrix problems}

To make the previous discussion concrete, suppose I want $y = Ax$, but
because of a small error in $A$ (due to measurement errors or roundoff
effects), I instead compute $\hat{y} = (A+E)x$ where $E$ is ``small.''
The expression for the {\em absolute} error is trivial:
\[
  \|\hat{y}-y\| = \|Ex\|.
\]
But I usually care more about the {\em relative error}.
\[
  \frac{\|\hat{y}-y\|}{\|y\|} = \frac{\|Ex\|}{\|y\|}.
\]
If we assume that $A$ is invertible and that we are using consistent
norms (which we will usually assume), then
\[
  \|Ex\| = \|E A^{-1} y\| \leq \|E\| \|A^{-1}\| \|y\|,
\]
which gives us
\[
  \frac{\|\hat{y}-y\|}{\|y\|} \leq \|A\| \|A^{-1}\|
  \frac{\|E\|}{\|A\|} = \kappa(A) \frac{\|E\|}{\|A\|}.
\]
That is, the relative error in the output is the relative error in the
input multiplied by the condition number
$\kappa(A) = \|A\| \|A^{-1}\|$.
%
Technically, this is the condition number for the problem of matrix
multiplication (or solving linear systems, as we will see) with
respect to a particular (consistent) norm; different problems have
different condition numbers.  Nonetheless, it is common to call this
``the'' condition number of $A$.

\section{Dimensions and scaling}

The first step in analyzing many application problems is
{\em nondimensionalization}: combining constants in the
problem to obtain a small number of dimensionless constants.
Examples include the aspect ratio of a rectangle,
the Reynolds number in fluid mechanics\footnote{%
Or any of a dozen other named numbers in fluid mechanics.  Fluid
mechanics is a field that appreciates the power of dimensional
analysis}, and so forth.  There are three big reasons to
nondimensionalize:
\begin{itemize}
\item
  Typically, the physics of a problem only really depends on
  dimensionless constants, of which there may be fewer than
  the number of dimensional constants.  This is important
  for parameter studies, for example.
\item
  For multi-dimensional problems in which the unknowns have different
  units, it is hard to judge an approximation error as ``small'' or
  ``large,'' even with a (normwise) relative error estimate.  But one
  can usually tell what is large or small in a non-dimensionalized
  problem.
\item
  Many physical problems have dimensionless parameters much less than
  one or much greater than one, and we can approximate the physics in
  these limits.  Often when dimensionless constants are huge or tiny
  and asymptotic approximations work well, naive numerical methods
  work work poorly.  Hence, nondimensionalization helps us choose how
  to analyze our problems --- and a purely numerical approach may be
  silly.
\end{itemize}

\section{Problems to ponder}

\begin{enumerate}
\item Show that as long as $\hat{e}_{\mathrm{rel}} < 1$,
  \[
  \frac{\hat{e}_{\mathrm{rel}}}{1+\hat{e}_{\mathrm{rel}}} \leq
  e_{\mathrm{rel}} \leq
  \frac{\hat{e}_{\mathrm{rel}}}{1-\hat{e}_{\mathrm{rel}}}.
  \]
\item Show that $A+E$ is invertible if $A$ is invertible and
  $\|E\| < 1/\|A^{-1}\|$ in some operator norm.
\item In this problem, we will walk through an argument about
  the bound on the relative error in approximating the
  relative error in solving a perturbed linear system:
  that is, how well does $\hat{y} = (A+E)^{-1} b$ approximate
  $y = A^{-1} b$ in a relative error sense?  We will assume
  throughout that $\|E\| < \epsilon$ and $\kappa(A) \epsilon < 1$.
  \begin{enumerate}
  \item Show that $\hat{y} = (I+A^{-1} E) y$.
  \item Using Neumann series bounds, argue that
    \[
      \|(I+A^{-1} E)-I\| \leq \frac{\|A^{-1} E\|}{1-\|A^{-1} E\|}
    \]
  \item Conclude that
    \[
    \frac{\|\hat{y}-y\|}{\|y\|} \leq
    \frac{\kappa(A) \epsilon}{1-\kappa(A) \epsilon}.
    \]
  \end{enumerate}
\end{enumerate}

\end{document}
