\documentclass[12pt, leqno]{article}
\usepackage{fancyhdr}
\usepackage[sort&compress]{natbib}
\usepackage[letterpaper=true,colorlinks=true,linkcolor=black]{hyperref}

\usepackage{amsfonts}
\usepackage{amsmath}
\usepackage{amssymb}
\usepackage{color}
\usepackage{tikz}
\usepackage{pgfplots}
\usepackage{listings}
%\usepackage{courier}
%\usepackage[utf8]{inputenc}
%\usepackage[russian]{babel}

\lstdefinelanguage{Julia}%
  {morekeywords={abstract,break,case,catch,const,continue,do,else,elseif,%
      end,export,false,for,function,immutable,import,importall,if,in,%
      macro,module,otherwise,quote,return,switch,true,try,type,typealias,%
      using,while},%
   sensitive=true,%
   alsoother={$},%
   morecomment=[l]\#,%
   morecomment=[n]{\#=}{=\#},%
   morestring=[s]{"}{"},%
   morestring=[m]{'}{'},%
}[keywords,comments,strings]%

\lstset{
  numbers=left,
  basicstyle=\ttfamily\footnotesize,
  numberstyle=\tiny\color{gray},
  stepnumber=1,
  numbersep=10pt,
}

\newcommand{\iu}{\ensuremath{\mathrm{i}}}
\newcommand{\bbR}{\mathbb{R}}
\newcommand{\bbC}{\mathbb{C}}
\newcommand{\calV}{\mathcal{V}}
\newcommand{\calE}{\mathcal{E}}
\newcommand{\calG}{\mathcal{G}}
\newcommand{\calW}{\mathcal{W}}
\newcommand{\calP}{\mathcal{P}}
\newcommand{\macheps}{\epsilon_{\mathrm{mach}}}
\newcommand{\matlab}{\textsc{Matlab}}

\newcommand{\ddiag}{\operatorname{diag}}
\newcommand{\fl}{\operatorname{fl}}
\newcommand{\nnz}{\operatorname{nnz}}
\newcommand{\tr}{\operatorname{tr}}
\renewcommand{\vec}{\operatorname{vec}}

\newcommand{\vertiii}[1]{{\left\vert\kern-0.25ex\left\vert\kern-0.25ex\left\vert #1
    \right\vert\kern-0.25ex\right\vert\kern-0.25ex\right\vert}}
\newcommand{\ip}[2]{\langle #1, #2 \rangle}
\newcommand{\ipx}[2]{\left\langle #1, #2 \right\rangle}
\newcommand{\order}[1]{O( #1 )}

\newcommand{\kron}{\otimes}


\newcommand{\hdr}[1]{
  \pagestyle{fancy}
  \lhead{Bindel, Spring 2020}
  \rhead{Numerical Analysis}
  \fancyfoot{}
  \begin{center}
    {\large{\bf #1}}
  \end{center}
  \lstset{language=Julia,columns=flexible}  
}


\begin{document}
\hdr{2020-02-12}

\section{Backward error in Gaussian elimination}

Solving $Ax = b$ in finite precision using Gaussian elimination
followed by forward and backward substitution yields a computed
solution $\hat{x}$ {\em exactly} satisfies
\begin{equation} \label{gauss-bnd}
  (A + \delta A) \hat{x} = b,
\end{equation}
where $|\delta A| \lesssim 3n\macheps |\hat{L}| |\hat{U}|$, assuming
$\hat{L}$ and $\hat{U}$ are the computed $L$ and $U$ factors.

I will now briefly sketch a part of the error analysis following
Demmel's treatment (\S 2.4.2, {\em Applied Numerical Linear Algebra}).
Here is the idea.  Suppose $\hat{L}$ and $\hat{U}$ are the computed
$L$ and $U$ factors.  We obtain $\hat{u}_{jk}$ by repeatedly
subtracting $l_{ji} u_{ik}$ from the original $a_{jk}$, i.e.
\[
  \hat{u}_{jk} =
    \fl\left( a_{jk} - \sum_{i=1}^{j-1} \hat{l}_{ji} \hat{u}_{ik} \right).
\]
Regardless of the order of the sum, we get an error that looks like
\[
  \hat{u}_{jk} = a_{jk}(1+\delta_0) -
                 \sum_{i=1}^{j-1} \hat{l}_{ji} \hat{u}_{ik} (1+\delta_i) +
                 O(\macheps^2)
\]
where $|\delta_i| \leq (j-1) \macheps$.  Turning this around gives
\begin{align*}
  a_{jk} & =
  \frac{1}{1+\delta_0} \left(
    \hat{l}_{jj} \hat{u}_{jk} +
    \sum_{i=1}^{j-1} \hat{l}_{ji} \hat{u}_{ik} (1 + \delta_i)
  \right) + O(\macheps^2) \\
  & =
    \hat{l}_{jj} \hat{u}_{jk} (1 - \delta_0) +
    \sum_{i=1}^{j-1} \hat{l}_{ji} \hat{u}_{ik} (1 + \delta_i - \delta_0) +
    O(\macheps^2) \\
  & =
    \left( \hat{L} \hat{U} \right)_{jk} + E_{jk},
\end{align*}
where
\[
   E_{jk} =
    -\hat{l}_{jj} \hat{u}_{jk} \delta_0 +
    \sum_{i=1}^{j-1} \hat{l}_{ji} \hat{u}_{ik} (\delta_i - \delta_0) +
    O(\macheps^2)
\]
is bounded in magnitude by $(j-1) \macheps (|L||U|)_{jk} + O(\macheps^2)$\footnote{
  It's obvious that $E_jk$ is bounded in magnitude by $2(j-1)
  \macheps(|L||U|)_{jk} + O(\macheps^2)$.  We cut a factor of two if
  we go down to the level of looking at the individual rounding errors
  during the dot product, because some of those errors cancel.
}.
A similar argument for the components of $\hat{L}$ yields
\[
  A = \hat{L} \hat{U} + E, \mbox{ where }
  |E| \leq n \macheps |\hat{L}| |\hat{U}| + O(\macheps^2).
\]

In addition to the backward error due to the computation of the $LU$
factors, there is also backward error in the forward and backward
substitution phases, which gives the overall bound (\ref{gauss-bnd}).

\section{Pivoting}

% The role of pivoting; partial pivoting

The backward error analysis in the previous section is not completely
satisfactory, since $|L| |U|$ may be much larger than $|A|$, yielding
a large backward error overall.  For example, consider the matrix
\[
  A = \begin{bmatrix} \delta & 1 \\ 1 & 1 \end{bmatrix} =
      \begin{bmatrix} 1 & 0 \\ \delta^{-1} & 1 \end{bmatrix}
      \begin{bmatrix} \delta & 1 \\ 0 & 1-\delta^{-1} \end{bmatrix}.
\]
If $0 < \delta \ll 1$ then $\|L\|_{\infty} \|U\|_{\infty} \approx
\delta^{-2}$, even though $\|A\|_{\infty} \approx 2$.  The problem is
that we ended up subtracting a huge multiple of the first row from the
second row because $\delta$ is close to zero --- that is, the leading
principle minor is {\em nearly} singular.  If $\delta$ were exactly
zero, then the factorization would fall apart even in exact
arithmetic.  The solution to the woes of singular and near singular minors
is pivoting; instead of solving a system with $A$, we re-order the
equations to get
\[
  \hat{A} =
      \begin{bmatrix} 1 & 1 \\ \delta & 1 \end{bmatrix} =
      \begin{bmatrix} 1 & 0 \\ \delta & 1 \end{bmatrix}
      \begin{bmatrix} 1 & 1 \\ 0 & 1-\delta \end{bmatrix}.
\]
Now the triangular factors for the re-ordered system matrix $\hat{A}$
have very modest norms, and so we are happy.  If we think of the re-ordering
as the effect of a permutation matrix $P$, we can write
\[
  A = \begin{bmatrix} \delta & 1 \\ 1 & 1 \end{bmatrix} =
      \begin{bmatrix} 0 & 1 \\ 1 & 0 \end{bmatrix}
      \begin{bmatrix} 1 & 0 \\ \delta & 1 \end{bmatrix}
      \begin{bmatrix} 1 & 1 \\ 0 & 1-\delta \end{bmatrix}
    = P^T LU.
\]
Note that this is equivalent to writing $P A = LU$ where $P$
is another permutation (which undoes the action of $P^T$).

If we wish to control the multipliers, it's natural to choose
the permutation $P$ so that each of the multipliers is at most one.
This standard choice leads to the following algorithm:
\begin{lstlisting}
  for j = 1:n-1

    % Find ipiv >= j to maximize |A(i,j)|
    [absAij, ipiv] = max(abs(A(j:n,j)));
    ipiv = ipiv + j-1;

    % Swap row ipiv and row j
    Aj = A(j,j:n);
    A(j,j:n) = A(ipiv,j:n);
    A(ipiv,j:n) = Aj;

    % Record the pivot row
    piv(j) = ipiv;

    % Update trailing submatrix
    A(j+1:n,j+1:n) = A(j+1:n,j+1:n) - A(j+1:n,j)*A(j,j+1:n);

  end
\end{lstlisting}

By design, this algorithm produces an $L$ factor in which all the
elements are bounded by one.  But what about the $U$ factor?  There
exist pathological matrices for which the elements of $U$ grow
exponentially with $n$.  But these examples are extremely uncommon in
practice, and so, in general, Gaussian elimination with partial
pivoting does indeed have a small backward error.  Of course, the
pivot growth is something that we can monitor, so in the unlikely event
that it {\em does} look like things are blowing up, we can tell there
is a problem and try something different.

When problems do occur, it is more frequently the result of
ill-conditioning in the problem than of pivot growth during the
factorization.

\section{Residuals revisited}

The analysis in the previous section is potentially pessimistic, and
does not cover all possible contingencies.  What if we use a solver
other than Gaussian elimination?  Will we have to completely redo our
error analysis?  If we know $A$ and $b$, a reasonable way to evaluate
an approximate solution $\hat{x}$ independent of how we got it is
through the residual $r = b-A\hat{x}$.  The approximate solution
satisfies
\[
  A \hat{x} = b + r,
\]
so if we subtract of $Ax = b$, we have
\[
  \hat{x}-x = A^{-1} r.
\]
We can use this to get the error estimate
\[
  \|\hat{x}-x\| = \|A^{-1}\| \|r\|;
\]
or we can get a more refined error estimate based on $\| \, |A^{-1}|
\, |r| \, \|$, as you will work out in the next homework.  But for a
given $\hat{x}$, we also actually have a prayer of {\em evaluating}
$\delta x = A^{-1} r$ with at least some accuracy.  This will be the
idea behind {\em iterative refinement}.

\section{Iterative refinement}

If we have a solver for $\hat{A} = A + E$ with $E$ small, then we can
use {\em iterative refinement} to ``clean up'' the solution.  The
matrix $\hat{A}$ could come from finite precision Gaussian elimination
of $A$, for example, or from some factorization of a nearby ``easier''
matrix.  To get the refinement iteration, we take the equation
\begin{equation} \label{fixedp}
  Ax = \hat{A}x-Ex = b,
\end{equation}
and think of $x$ as the fixed point for an iteration
\begin{equation} \label{itref-fixedp}
  \hat{A} x_{k+1} - E x_k = b.
\end{equation}
Note that this is the same as
\[
  \hat{A} x_{k+1} - (\hat{A} - A) x_k = b,
\]
or
\[
  x_{k+1} = x_k + \hat{A}^{-1} (b - A x_{k}).
\]
Note that this latter form is the same as inexact Newton iteration on
the equation $A x_{k} - b = 0$ with the approximate Jacobian $\hat{A}$.

If we subtract (\ref{fixedp}) from (\ref{itref-fixedp}), we see
\[
  \hat{A}(x_{k+1}-x) - E(x_k-x) = 0,
\]
or
\[
  x_{k+1}-x = \hat{A}^{-1} E (x_k-x).
\]
Taking norms, we have
\[
  \|x_{k+1}-x\| \leq \|\hat{A}^{-1} E\| \|x_k-x\|.
\]
Thus, if $\|\hat{A}^{-1} E\| < 1$, we are guaranteed that $x_{k} \rightarrow x$
as $k \rightarrow \infty$.  At least, this is what happens in exact arithmetic.
In practice, the residual is usually computed with only finite precision,
and so we would stop making progress at some point.  In general,
iterative refinement is mainly used when either the residual can be
computed with extra precision or when the original solver suffers from
relatively large backward error.

\end{document}
