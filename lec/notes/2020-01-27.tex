\documentclass[12pt, leqno]{article}
\usepackage{fancyhdr}
\usepackage[sort&compress]{natbib}
\usepackage[letterpaper=true,colorlinks=true,linkcolor=black]{hyperref}

\usepackage{amsfonts}
\usepackage{amsmath}
\usepackage{amssymb}
\usepackage{color}
\usepackage{tikz}
\usepackage{pgfplots}
\usepackage{listings}
%\usepackage{courier}
%\usepackage[utf8]{inputenc}
%\usepackage[russian]{babel}

\lstdefinelanguage{Julia}%
  {morekeywords={abstract,break,case,catch,const,continue,do,else,elseif,%
      end,export,false,for,function,immutable,import,importall,if,in,%
      macro,module,otherwise,quote,return,switch,true,try,type,typealias,%
      using,while},%
   sensitive=true,%
   alsoother={$},%
   morecomment=[l]\#,%
   morecomment=[n]{\#=}{=\#},%
   morestring=[s]{"}{"},%
   morestring=[m]{'}{'},%
}[keywords,comments,strings]%

\lstset{
  numbers=left,
  basicstyle=\ttfamily\footnotesize,
  numberstyle=\tiny\color{gray},
  stepnumber=1,
  numbersep=10pt,
}

\newcommand{\iu}{\ensuremath{\mathrm{i}}}
\newcommand{\bbR}{\mathbb{R}}
\newcommand{\bbC}{\mathbb{C}}
\newcommand{\calV}{\mathcal{V}}
\newcommand{\calE}{\mathcal{E}}
\newcommand{\calG}{\mathcal{G}}
\newcommand{\calW}{\mathcal{W}}
\newcommand{\calP}{\mathcal{P}}
\newcommand{\macheps}{\epsilon_{\mathrm{mach}}}
\newcommand{\matlab}{\textsc{Matlab}}

\newcommand{\ddiag}{\operatorname{diag}}
\newcommand{\fl}{\operatorname{fl}}
\newcommand{\nnz}{\operatorname{nnz}}
\newcommand{\tr}{\operatorname{tr}}
\renewcommand{\vec}{\operatorname{vec}}

\newcommand{\vertiii}[1]{{\left\vert\kern-0.25ex\left\vert\kern-0.25ex\left\vert #1
    \right\vert\kern-0.25ex\right\vert\kern-0.25ex\right\vert}}
\newcommand{\ip}[2]{\langle #1, #2 \rangle}
\newcommand{\ipx}[2]{\left\langle #1, #2 \right\rangle}
\newcommand{\order}[1]{O( #1 )}

\newcommand{\kron}{\otimes}


\newcommand{\hdr}[1]{
  \pagestyle{fancy}
  \lhead{Bindel, Spring 2020}
  \rhead{Numerical Analysis}
  \fancyfoot{}
  \begin{center}
    {\large{\bf #1}}
  \end{center}
  \lstset{language=Julia,columns=flexible}  
}


\begin{document}
\hdr{2020-01-27}

Most of mathematics is best learned by doing.  Linear algebra is no
exception.  You have had a previous class in which you learned the
basics of linear algebra, and you will have plenty of practice with
these concepts over the semester.  This brief refresher lecture is
supposed to remind you of what you've already learned and introduce a
few things you may not have seen.  It also serves to set notation
that will be used throughout the class.

In addition to these notes, you may find it useful to go back to a
good linear algebra text (there are several listed on the course
syllabus) or to look at the linear algebra review chapters in the
book.

\section{Vectors}

A vector space (or linear space) is a set of vectors that can be added
or scaled in a sensible way -- that is, addition is associative and
commutative and scaling is distributive.  We will generally denote
vector spaces by script letters (e.g.~$\mathcal{V}, \mathcal{W}$),
vectors by lower case Roman letters (e.g.~$v, w$), and scalars by
lower case Greek letters (e.g.~$\alpha, \beta$).  But we feel free to
violate these conventions according to the dictates of our conscience
or in deference to other conflicting conventions.

There are many types of vector spaces.  Apart from the ubiquitous
spaces $\bbR^n$ and $\bbC^n$, the most common vector spaces in applied
mathematics are different types of function spaces.  These include
\begin{align*}
  \mathcal{P}_d &=
  \{ \mbox{polynomials of degree at most $d$} \}; \\
  \mathcal{V}^* &=
  \{ \mbox{linear functions $\mathcal{V} \rightarrow \bbR$ (or $\bbC$)} \}; \\
  L(\mathcal{V}, \mathcal{W}) &=
  \{ \mbox{linear maps $\calV \rightarrow \calW$} \}; \\
  \mathcal{C}^k(\Omega) &=
  \{\mbox{ $k$-times differentiable functions on a set $\Omega$} \};
\end{align*}
and many more.  We compute with vectors in $\bbR^n$ and $\bbC^n$,
which we represent concretely by tuples of numbers in memory, usually
stored in sequence.  To keep a broader perspective, though, we will
also frequently describe examples involving the polynomial spaces
$\mathcal{P}_d$.

\subsection{Spanning sets and bases}

We often think of a matrix as a set of vectors
\[
  V = \begin{bmatrix} v_1 & v_2 & \ldots & v_n \end{bmatrix}.
\]
The {\em range space} $\mathcal{R}(V)$ or the {\em span}
$\operatorname{sp}\{ v_j \}_{j=1}^n$ is the set of vectors
$\{ V c : c \in \bbR^n \}$ (or $\bbC^n$
for complex spaces).  The vectors are {\em linearly independent}
if any vector in the span has a unique representation as a linear
combination of the spanning vectors; equivalently, the vectors are
linearly independent if there is no linear combination of the
spanning vectors that equals zero except the one in which all
coefficients are zero.  A linearly independent set of vectors is
a {\em basis} for a space $\mathcal{V}$ if
$\operatorname{sp}\{v_j\}_{j=1}^n = \mathcal{V}$; the number of
basis vectors $n$ is the {\em dimension} of the space.  Spaces
like $\mathcal{C}^k(\Omega)$ do not generally have a finite basis;
they are {\em infinite-dimensional}.  We will focus on
finite-dimensional spaces in the class, but it is useful to know
that there are interesting infinite-dimensional spaces in the
broader world.

The {\em standard basis in $\bbR^n$} is the set of column vectors
$e_j$ with a one in the $j$th position and zeros elsewhere:
\[
  I = \begin{bmatrix} e_1 & e_2 & \ldots & e_n \end{bmatrix}.
\]
Of course, this is not the only basis.  Any other set of $n$ linearly
independent vectors in $\bbR^n$ is also a basis
\[
  V = \begin{bmatrix} v_1 & v_2 & \ldots & v_n \end{bmatrix}.
\]
The matrix $V$ formed in this way is invertible, with multiplication
by $V$ corresponding to a change from the $\{v_j\}$ basis into the
standard basis and multiplication by $V^{-1}$ corresponding to a map
from the $\{v_j\}$ basis to the standard basis.

We will use matrix-like notation to describe bases and spanning sets
even when we deal with spaces other than $\bbR^n$.  For example, a
common basis for $\mathcal{P}_d$ is the {\em power basis}
\[
  X = \begin{bmatrix} 1 & x & x^2 & \ldots & x^d \end{bmatrix}.
\]
Each ``column'' in $X$ is really a function of one variable ($x$), and
matrix-vector multiplication with $X$ represents a map from a
coefficient vector in $\bbR^{d+1}$ to a polynomial in $\mathcal{P}_d$.
That is, we write polynomials $p \in \mathcal{P}_d$ in terms of the
basis as
\[
  p = Xc = \sum_{j=0}^d c_j x^j
\]
and, we think of computing the coefficients from the
abstract polynomial via a formal inverse:
\[
  c = X^{-1} p.
\]
We typically think of a map like $Y^* = X^{-1}$ in terms of ``rows''
\[
  Y^* = \begin{bmatrix} y_0^* \\ y_1^* \\ \vdots \\ y_d^* \end{bmatrix}
\]
where each row $y_j^*$ is a {\em linear functional} or {\em dual
  vector} (i.e.~linear mappings from the vector space to the real or
complex numbers).  Collectively, $\{y_0^*, \ldots, y_d^*\}$ are the
{\em dual basis} to $\{1, x, \ldots, x^d\}$.

The power basis is not the only basis for $\mathcal{P}_d$.  Other
common choices include Newton or Lagrange polynomials with respect to
a set of points, which you may have seen in another class such as CS
4210.  In this class, we will sometimes use the Chebyshev\footnote{
  Pafnuty Chebyshev (Чебышёв) was a nineteenth century Russian
  mathematician, and his name has been transliterated from the
  Cyrillic alphabet into the Latin alphabet in several different ways.
  We inherit our usual spelling from one of the French
  transliterations, but the symbol $T$ for the polynomials comes from
  the German transliteration Tschebyscheff.  } polynomial basis $\{
T_j(x) \}$ given by the recurrence
\begin{align*}
  T_0(x) &= 1 \\
  T_1(x) &= x \\
  T_{j+1}(x) &= 2 x T_{j}(x) - T_{j-1}(x), \quad j \geq 1,
\end{align*}
and the Legendre polynomial basis
$\{ P_j(x) \}$, given by the recurrence
\begin{align*}
  P_0(x) &= 1 \\
  P_1(x) &= x \\
  (j+1) P_{j+1}(x) &= (2j+1) x P_j(x) - j P_{j-1}(x).
\end{align*}
As we will see over the course of the semester, sometimes the
``obvious'' choice of basis (e.g.~the standard basis in $\bbR^n$ or
the power basis in $\mathcal{P}_d$) is not the best choice for
numerical computations.

\subsection{Vector norms}

A {\em norm} $\|\cdot\|$ measures vector lengths.  It is
positive definite, homogeneous, and sub-additive:
\begin{align*}
  \|v\| & \geq 0 \mbox{ and } \|v\| = 0 \mbox{ iff } v = 0 \\
  \|\alpha v\| &= |\alpha| \|v\| \\
  \|u+v\| & \leq \|u\| + \|v\|.
\end{align*}
The three most common vector norms we work with in $\bbR^n$ are the
Euclidean norm (aka the 2-norm), the $\infty$-norm (or max norm),
and the $1$-norm:
\begin{align*}
  \|v\|_2 &= \sqrt{\sum_j |v_j|^2} \\
  \|v\|_\infty &= \max_j |v_j| \\
  \|v\|_1 &= \sum_j |v_j|
\end{align*}

Many other norms can be related to one of these three norms.  In
particular, a ``natural'' norm in an abstract vector space will often
look strange in the corresponding concrete representation with respect
to some basis function.  For example, consider the vector space of
polynomials with degree at most 2 on $[-1,1]$.  This space also has
a natural Euclidean norm, max norm, and $1$-norm; for a given
polynomial $p(x)$ these are
\begin{align*}
  \|p\|_2 &= \sqrt{ \int_{-1}^1 |p(x)|^2 \, dx } \\
  \|p\|_\infty &= \max_{x \in [-1,1]} |p(x)| \\
  \|p\|_1 &= \int_{-1}^1 |p(x)| \, dx.
\end{align*}
But when we write $p(x)$ in terms of the coefficient vector with
respect to the power basis (for example), the max norm of the
polynomial is not the same as the max norm of the coefficient vector.
In fact, if we consider a polynomial $p(x) = c_0 + c_1 x$, then
the max norm of the polynomial $p$ is the same as the one-norm of the
coefficient vector --- the proof of which is left as a useful exercise
to the reader.

In a finite-dimensional vector space, all norms are {\em equivalent}:
that is, if $\|\cdot\|$ and $\vertiii{\cdot}$ are two norms on the
same finite-dimensional vector space, then there exist constants $c$
and $C$ such that for any $v$ in the space,
\[
  c \|v\| \leq \vertiii{v} \leq C \|v\|.
\]
Of course, there is no guarantee that the constants are small!

An {\em isometry} is a mapping that preserves vector norms.
For $\bbR^n$, the only isometries for the 1-norm and the $\infty$-norm
are permutations.  For Euclidean space, though, there is a much
richer set of isometries, represented by the orthogonal matrices
(matrices s.t.~$Q^* Q = I$).

% Vector spaces
% - Abstract picture
% - R^n
% - Polynomials
% - Norms

\subsection{Inner products}

An {\em inner product} $\langle \cdot, \cdot \rangle$
is a function from two vectors into the real
numbers (or complex numbers for an complex vector space).  It is
positive definite, linear in the first slot, and symmetric (or
Hermitian in the case of complex vectors); that is:
\begin{align*}
  \langle v, v \rangle & \geq 0 \mbox{ and }
  \langle v, v \rangle = 0 \mbox{ iff } v = 0 \\
%
  \langle \alpha u, w \rangle &= \alpha \langle u, w \rangle
  \mbox{ and }
  \langle u+v, w \rangle = \langle u, w \rangle + \langle v, w \rangle \\
%
  \langle u, v \rangle &= \overline{\langle v, u \rangle},
\end{align*}
where the overbar in the latter case corresponds to complex
conjugation.  A vector space with an inner product is sometimes
called an {\em inner product space} or a {\em Euclidean space}.

Every inner product defines a corresponding norm
\[
  \|v\| = \sqrt{ \langle v, v \rangle}
\]
The inner product and the associated norm satisfy
the {\em Cauchy-Schwarz} inequality
\[
  \langle u, v \rangle \leq \|u\| \|v\|.
\]
The {\em standard inner product} on $\bbC^n$ is
\[
  x \cdot y = y^* x = \sum_{j=1}^n \bar{y}_j x_j.
\]
But the standard inner product is not the only inner product,
just as the standard Euclidean norm is not the only norm.

Just as norms allow us to reason about size, inner products let us
reason about angles.  In particular, we define the cosine of
the angle $\theta$ between nonzero vectors $v$ and $w$ as
\[
  \cos \theta = \frac{\langle v, w \rangle}{\|v\| \|w\|}.
\]

Returning to our example of a vector space of polynomials,
the standard $L^2([-1,1])$ inner product is
\[
  \langle p, q \rangle_{L^2([-1,1])} = \int_{-1}^1 p(x) \bar{q}(x) \, dx.
\]
If we express $p$ and $q$ with respect to a basis (e.g.~the power
basis), we fine that we can represent this inner product via a
symmetric positive definite matrix.  For example,
let $p(x) = c_0 + c_1 x + c_2 x^2$ and let
$q(x) = d_0 + d_1 x + d_2 x^2$.  Then
\[
  \langle p, q \rangle_{L^2([-1,1])} =
  \begin{bmatrix} d_0 \\ d_1 \\ d_2 \end{bmatrix}^*
  \begin{bmatrix}
    a_{00} & a_{01} & a_{02} \\
    a_{10} & a_{11} & a_{12} \\
    a_{20} & a_{21} & a_{22}
  \end{bmatrix}
  \begin{bmatrix} c_0 \\ c_1 \\ c_2 \end{bmatrix} =
  d^* A c = \langle c, d \rangle_A
\]
where
\[
a_{ij} = \int_{-1}^1 x^{i-1} x^{j-1} \, dx =
\begin{cases}
  2/(i+j-1), & i+j \mbox{ even} \\
  0, & \mbox{ otherwise}
\end{cases}
\]
The symmetric positive definite matrix $A$ is what is sometimes called
the {\em Gram matrix} for the basis $\{1, x, x^2\}$.

We say two vectors $u$ and $v$ are {\em orthogonal} with respect to an
inner product if $\langle u, v \rangle = 0$.  If $u$ and $v$ are
orthogonal, we have the {\em Pythagorean theorem}:
\begin{align*}
  \|u+v\|^2
  = \langle u + v, u + v \rangle
  = \langle u, u \rangle + \langle v, u \rangle + \langle u, v \rangle
  + \langle v, v \rangle
  = \|u\|^2 + \|v\|^2
\end{align*}
Two vectors $u$ and $v$ are {\em orthonormal} if they are orthogonal
with respect to the inner product and have unit length in the associated
norm.  When we work in an inner product space, we often use an
{\em orthonormal basis}, i.e.~a basis in which all the vectors are
orthonormal.  For example, the normalized Legendre polynomials
\[
  \sqrt{\frac{2}{2j+1}} P_j(x)
\]
form orthonormal bases for the $\mathcal{P}_d$ inner product with
respect to the $L^2([-1,1])$ inner product.

\section{Matrices and mappings}

A matrix represents a mapping between two vector spaces.  That is, if
$L : \mathcal{V} \rightarrow \mathcal{W}$ is a linear map, then the
associated matrix $A$ with respect to bases $V$ and $W$ satisfies
$A = W^{-1} L V$.  The same linear mapping corresponds to different
matrices depending on the choices of basis.  But matrices can reresent
several other types of mappings as well.  Over the course of this
class, we will see several interpretations of matrices:
\begin{itemize}
\item {\bf Linear maps}.  A map $L : \calV \rightarrow \calW$ is
  linear if $L(x+y) = Lx + Ly$ and $L(\alpha x) = \alpha Lx$.
  The corresponding matrix is $A = W^{-1} L V$.
\item {\bf Linear operators}.  A linear map from a space to itself
  ($L : \calV \rightarrow \calV$) is a linear operator.
  The corresponding (square) matrix is $A = V^{-1} L V$.
\item {\bf Bilinear forms}.  A map
  $a : \calV \times \calW \rightarrow \bbR$ (or $\bbC$ for complex
  spaces) is bilinear if it is linear in both slots:
  $a(\alpha u+v, w) = \alpha a(u,w) + a(v,w)$ and
  $a(v, \alpha u + w) = \alpha a(v,u) + a(v,w)$.
  The corresponding matrix has elements $A_{ij} = a(v_i, w_j)$;
  if $v = Vc$ and $w = Wd$ then $a(v,w) = d^T A c$.

  We call a bilinear form on $\calV \times \calV$ {\em symmetric} if
  $a(v,w) = a(w,v)$; in this case, the corresponding matrix $A$ is
  also symmetric ($A = A^T$).  A symmetric form and the corresponding
  matrix are called {\em positive semi-definite} if $a(v,v) \geq 0$
  for all $v$.  The form and matrix are {\em positive definite} if
  $a(v,v) > 0$ for any $v \neq 0$.

  A {\em skew-symmetric} matrix ($A = -A^T$) corresponds to a
  skew-symmetric or anti-symmetric bilinear form,
  i.e.~$a(v,w) = -a(w,v)$.
\item {\bf Sesquilinear forms}.  A map
  $a : \calV \times \calW \rightarrow \bbC$
  (where $\calV$ and $\calW$ are complex vector
  spaces) is sesquilinear if it is linear in the first slot and
  the conjugate is linear in the second slot:
  $a(\alpha u+v, w) = \alpha a(u,w) + a(v,w)$ and
  $a(v, \alpha u + w) = \bar{\alpha} a(v,u) + a(v,w)$.
  The matrix has elements $A_{ij} = a(v_i, w_j)$;
  if $v = Vc$ and $w = Wd$ then $a(v,w) = d^* A c$.

  We call a sesquilinear form on $\calV \times \calV$ {\em Hermitian} if
  $a(v,w) = a(w,v)$; in this case, the corresponding matrix $A$ is
  also Hermitian ($A = A^*$).  A Hermitian form and the corresponding
  matrix are called {\em positive semi-definite} if $a(v,v) \geq 0$
  for all $v$.  The form and matrix are {\em positive definite} if
  $a(v,v) > 0$ for any $v \neq 0$.

  A {\em skew-Hermitian} matrix
  ($A = -A^*$) corresponds to a skew-Hermitian or anti-Hermitian bilinear
  form, i.e.~$a(v,w) = -a(w,v)$.
\item {\bf Quadratic forms}.  A quadratic form $\phi : \calV
  \rightarrow \bbR$ (or $\bbC$) is a homogeneous quadratic function
  on $\calV$, i.e.~$\phi(\alpha v) = \alpha^2 \phi(v)$ for which the
  map $b(v,w) = \phi(v+w) - \phi(v) - \phi(w)$ is bilinear.
  Any quadratic form on a finite-dimensional space can be
  represented as $c^* A c$ where $c$ is the coefficient vector for
  some Hermitian matrix $A$.  The formula for the elements of $A$
  given $\phi$ is left as an exercise.
\end{itemize}
We care about linear maps and linear operators almost everywhere, and
most students come out of a first linear algebra class with some
notion that these are important.  But apart from very standard
examples (inner products and norms), many students have only a vague
notion of what a bilinear form, sesquilinear form, or quadratic form
might be.  Bilinear forms and sesquilinear forms show up when we
discuss large-scale solvers based on projection methods.  Quadratic
forms are important in optimization, physics (where they often
represent energy), and statistics (e.g.~for understanding variance and
covariance).

\subsection{Matrix norms}

The space of matrices forms a vector space; and, as with other vector
spaces, it makes sense to talk about norms.  In particular, we
frequently want norms that are {\em consistent} with vector norms
on the range and domain spaces; that is, for any $w$ and $v$,
we want
\[
  w = Av \implies \|w\| \leq \|A\| \|v\|.
\]
One ``obvious'' consistent norm is the {\em Frobenius norm},
\[
  \|A\|_F^2 = \sum_{i,j} a_{ij}^2.
\]
Even more useful are {\em induced norms} (or {\em operator norms})
\[
  \|A\| = \sup_{v \neq 0} \frac{\|Av\|}{\|v\|} = \sup_{\|v\|=1} \|Av\|.
\]
The induced norms corresponding to the vector 1-norm and $\infty$-norm
are
\begin{align*}
  \|A\|_1 &= \max_j \sum_i |a_{ij}| \quad \mbox{(max column sum)}\\
  \|A\|_\infty &= \max_i \sum_j |a_{ij}| \quad \mbox{(max row sum)}
\end{align*}
The norm induced by the vector Euclidean norm (variously called
the matrix 2-norm or the spectral norm) is more complicated.

The Frobenius norm and the matrix 2-norm are both {\em orthogonally
  invariant} (or {\em unitarily invariant} in a complex vector space.
That is, if $Q$ is a square matrix with $Q^* = Q^{-1}$ (an orthogonal
or unitary matrix) of the appropriate dimensions
\begin{align*}
  \|QA\|_F &= \|A\|_F, &
  \|AQ\|_F &= \|A\|_F, \\
  \|QA\|_2 &= \|A\|_2, &
  \|AQ\|_2 &= \|A\|_2.
\end{align*}
This property will turn out to be frequently useful throughout the course.

\subsection{Decompositions and canonical forms}

{\em Matrix decompositions} (also known as
{\em matrix factorizations}) are central to numerical linear algebra.
We will get to know six such factorizations well:
\begin{itemize}
\item
  $PA = LU$ (a.k.a.~Gaussian elimination).  Here $L$ is unit lower
  triangular (triangular with 1 along the main diagonal), $U$ is upper
  triangular, and $P$ is a permutation matrix.
\item
  $A = LL^*$ (a.k.a.~Cholesky factorization).  Here $A$ is Hermitian
  and positive definite, and $L$ is a lower triangular matrix.
\item
  $A = QR$ (a.k.a.~QR decomposition).  Here $Q$ has orthonormal
  columns and $R$ is upper triangular.  If we think of the columns
  of $A$ as a basis, QR decomposition corresponds to the Gram-Schmidt
  orthogonalization process you have likely seen in the past (though
  we rarely compute with Gram-Schmidt).
\item
  $A = U \Sigma V^*$ (a.k.a.~the singular value decomposition or SVD).
  Here $U$ and $V$ have orthonormal columns and $\Sigma$ is diagonal
  with non-negative entries.
\item
  $A = Q \Lambda Q^*$ (a.k.a.~symmetric eigendecomposition).  Here $A$
  is Hermitian (symmetric in the real case), $Q$ is orthogonal or
  unitary, and $\Lambda$ is a diagonal matrix with real numbers on the
  diagonal.
\item
  $A = QTQ^*$ (a.k.a.~Schur form).  Here $A$ is a square matrix, $Q$
  is orthogonal or unitary, and $T$ is upper triangular (or nearly
  so).
\end{itemize}

The last three of these decompositions correspond to
{\em canonical forms} for abstract operators.  That is, we can view
these decompositions as finding bases in which the matrix
representation of some operator or form is particularly simple.
More particularly:
\begin{itemize}
\item {\bf SVD}:
  For any linear mapping $L : \mathcal{V} \rightarrow \mathcal{W}$,
  there are orthonormal bases for the two spaces such that the
  corresponding matrix is diagonal
\item {\bf Symmetric eigendecomposition}:
  For any Hermitian sesquilinear map on an inner product space, there
  is an orthonormal basis for the space such that the matrix
  representation is diagonal.
\item {\bf Schur form}:
  For any linear operator $L : \mathcal{V} \rightarrow \mathcal{V}$,
  there is an orthonormal basis for the space such that the matrix
  representation is upper triangular.  Equivalently, if
  $\{u_1, \ldots, u_n\}$ is the basis in question,
  then $\operatorname{sp}(\{u_j\}_{j=1}^k)$ is an
  {\em invariant subspace} for each $1 \leq k \leq n$.
\end{itemize}
The Schur form turns out to be better for numerical work than the
Jordan canonical form that you should have seen in an earlier class.
We will discuss this in more detail when we discuss eigenvalue
problems.

\subsection{The SVD and the 2-norm}

The singular value decomposition is useful for a variety of reasons;
we close off the lecture by showing one such use.

Suppose $A = U \Sigma V^*$ is the singular value decomposition of some
matrix.  Using orthogonal invariance (unitary invariance) of the
2-norm, we have
\[
  \|A\|_2 = \|U^* A V\|_2 = \|\Sigma_2\|,
\]
i.e.~
\[
  \|A\|_2 = \max_{\|v\|^2 = 1} \frac{\sum_j \sigma_j |v_j|^2}{\sum |v_j|^2}.
\]
That is, the spectral norm is the largest weighted average of the
singular values, which is the same as just the largest singular value.

The small singular values also have a meaning.  If $A$ is a square,
invertible matrix then
\[
  \|A^{-1}\|_2 = \|V \Sigma^{-1} U^*\|_2 = \|\Sigma_{-1}\|_2,
\]
i.e.~$\|A^{-1}|_2$ is the inverse of the smallest singular value of $A$.

The smallest singular value of a nonsingular matrix $A$ can also be
interpreted as the ``distance to singularity'': if $\sigma_n$ is the
smallest singular value of $A$, then there is a matrix $E$ such that
$\|E\|_2 = \sigma_n$ and $A+E$ is singular; and there is no such
matrix with smaller norm.

These facts about the singular value decomposition are worth
pondering, as they will be particularly useful in the next lecture
when we ponder sensitivity and conditioning.

\section{Problems to ponder}

\begin{enumerate}
\item In terms of the power basis, write the matrix
  $A \in \bbR^{4 \times 3}$ corresponding to the mapping from
  $\mathcal{P}_2 \rightarrow \mathcal{P}_3$ given by multiplication by $x$.
\item Repeat the first problem in terms of the basis of Legendre
  polynomials.
\item What is the spectral norm of the operator $p \mapsto xp$ from
  $\mathcal{P}_2 \rightarrow \mathcal{P}_3$ with respect to the
  $L^2([-1,1])$ inner product and norm? Note that it may be
  helpful to write the matrix in terms of the normalized Legendre
  polynomials and to use MATLAB's {\tt svd} command.
\item What is the $L^{\infty}([-1,1])$ norm of $p \mapsto xp$ from
  $\mathcal{P}_2 \rightarrow \mathcal{P}_3$?  That is, we want
  the norm induced by the vector norm $\|p\| = \max_{x\in[-1,1]} |p(x)|$.
\item Show that if $p(x) = c_0 + c_1 x \in \mathcal{P}_2$, then
  $\|p\|_{\infty} = \|c\|_1$, where the first norm is a function
  max norm and the second is a norm on the coefficient vector.
\item Show the law of cosines in an inner product space:
  for nonzero vectors $u$ and $v$,
  \[
    \|u + v\|^2 = \|u\|^2 + \|v\|^2 + 2 \cos{\theta} \|u\| \|v\|
  \]
  where $\cos(\theta)$ is the angle between $u$ and $v$.
\item Show that the $1$-norm and $\infty$-norm are equivalent,
  i.e. for all $v \in \bbR^n$,
  \[
    c \|v\|_\infty \leq \|v\|_1 \leq C \|v\|_\infty
  \]
  What are the tightest possible constants $c$ and $C$?
\item Write the dual basis for the power basis $\{1, x, x^2\}$ for
  $\mathcal{P}_2$ in terms of function evaluations --- for example,
  $y_0^* p = p(0)$ gives the constant coefficient.
\end{enumerate}

\end{document}
